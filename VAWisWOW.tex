% fancytikzposter.tex, version 2.1
% Original template created by Elena Botoeva [botoeva@inf.unibz.it], June 2012
% 
% This file is distributed under the Creative Commons Attribution-NonCommercial 2.0
% Generic (CC BY-NC 2.0) license
% http://creativecommons.org/licenses/by-nc/2.0/ 
\documentclass{a0poster}
\usepackage[USenglish]{babel}% or british
\usepackage[T1]{fontenc}


\usepackage{wrapfig}
\usepackage{fancytikzposterM} 
\usepackage{multirow}
\usepackage{array}

% background picture
\usebackgroundtemplate{7}

\definecolor{myblue3}{HTML}{0C3E07}%main frames


\usepackage[margin=\margin cm, paperwidth=84.1cm, paperheight=118.9cm]{geometry}


\usepackage{cmbright}
%\usepackage[default]{cantarell}
%\usepackage{avant}
%\usepackage[math]{iwona}
\usepackage[math]{kurier}
\usepackage[T1]{fontenc}
\usepackage{tikz}

%%%%%%%%%%%%%%%%%%%%%%%%%%
%%% SET FONT SIZES %%%%%%%%%%%%%
%%%%%%%%%%%%%%%%%%%%%%%%%%
\renewcommand{\Huge}{\fontsize{64}{85}\selectfont} % Main title
\renewcommand{\huge}{\fontsize{44}{70}\selectfont} % Sub title
\renewcommand{\LARGE}{\fontsize{48}{48}\selectfont} % Box headers
\renewcommand{\Large}{\fontsize{42}{40}\selectfont} % Authors
\renewcommand{\normalsize}{\fontsize{40}{42}\selectfont} % Text in the boxes
\renewcommand{\footnotesize}{\fontsize{34}{42}\selectfont} % references
\renewcommand{\small}{\fontsize{24}{22}\selectfont}
%% add your packages here


%% Set the folder that contains the images
\graphicspath{ {./Lowres/} }


\title{How much are wild vertebrate populations evolving right now?\vspace{-0.2em}}
\author{{\LARGE{Timoth\'{e}e Bonnet} \& Loeske Kruuk}\\ \vspace{-20pt} \footnotesize Australian National University
\\
}

\begin{document}

%%%%% ---------- the background picture ---------- %%%%%
%% to change it modify the macro \BackgroundPicture
%\ClearShipoutPicture
%\AddToShipoutPicture{\BackgroundPicture}

\noindent % to have the picture right in the center

\begin{tikzpicture}
  \initializesizeandshifts
  % \setxshift{15}
  \setyshift{2.5} % uncomment this line to condense the boxes

  \newcommand{\volesize}{0.8cm}

  %% the title block, #1 - shift, the default value is (0,0), #2 - width, #3 - scale
  %% the alias of the title block is `title', so we can refer to its boundaries later
  \ifthenelse{\equal{\template}{1}}{ 
    \titleblock[(-0.7,0)]{48}{1}
			}{
    \titleblock[(-10,0)]{87}{1.5}
			}
			
\addlogo[east]{(10,0)}{8cm}{anu}
\addlogo[west]{(-10,0)}{8cm}{tim}

\coordinate (aaa) at (currenty);

	\blocknodew[($(currenty)+(19.6,0)$)]{77}{\textsc{The big problem:} We do not know how much wild organisms are currently evolving!}
		{Fisher's fundamental theorem of natural selection states that \textbf{additive genetic variation in fitness measures evolution across all traits and all the genome}. That is just what we need{\color{colorone}{\**}}!
		Yet, there are few estimates in free-ranging populations, and most may be unreliable. Indeed, it is difficult to measure fitness, difficult to estimate genetic variance, statistical models tend not to fit the data, and it is unclear how to interpret estimates from generalized linear models. We assemble data from the monitoring of a dozen pedigreed populations, 
		}
	
%%%%%%%%%%%%%%%%%%%%%%%%%%%%%%%%%%%%%%%%%%%%%%%%%%%%%%%%%%%%%%%%%%%%%%%%%%%%%	
	\blocknodew[($(currenty)+(0,2)$)]{77}{\textsc{Theory:} How to estimate additive genetic variance in relative fitness ($V_A(\omega)$) }
	{
	\hspace{-1cm}
		\begin{tabular}{p{0.30\textwidth} p{0.03\textwidth} p{0.30\textwidth} p{0.03\textwidth} p{0.30\textwidth}}
		
		\parbox{25cm}{model fitness?\\
		  \centering  \begin{tabular}{c c}

		    \end{tabular}
		  }
		&
		  ZI distribution
		&
		\parbox{25cm}{ \hspace{4cm}estimate genetic variation?\\
		   \centering \begin{tabular}{c} \end{tabular}
		  }
		&
		  
		&
		\parbox{25cm}{Convert estimates to $V_A(\omega)$\\
		\centering \begin{tabular}{c}  \end{tabular}
		  }	
		\end{tabular}
		\vspace{-1cm}
	}%end block
%%%%%%%%%%%%%%%%%%%%%%%%%%%%%%%%%%%%%%%%%%%%%%%%%%%%%%%%%%%%%%%%%%%%%%%%%%%%%	
	\coordinate (tfirstcol) at (currenty);
	
 \coordinate (width) at ($(76.8,0) - (2.1,0)$);   
 %% the content of the block
          \draw let \p1=($(width)-(0,0)$) , \p2=($(0,\blocktitleheight cm)-(0,2.4cm)$)
          in node[draw, anchor=north, color=blocktitlefillcolor, text=blocktextcolor, 
          frame, rectanglesplittwo, rectangle split horizontal=false, 
          rectangle split part fill={blocktitlefillcolor, none}, %
          rectangle split empty part height= \y2 
          ]
          (box) at ($(currenty)+(0,2)$) {
            \nodepart{second}
            \begin{minipage}{\x1}
            \begin{center}
	      \vspace{22cm} % a very mesy trick to control box height...
            \end{center}
           \end{minipage}
          }; 

          %% the title of the block
          \node[frame, anchor=north west, text=blocktitletextcolor] at (box.north west)
          {\bf\LARGE \textsc{Emperical results:} };
          
   \coordinate (currenty) at ($(box.south)-(yshift)$);
%%%%%%%%%%%%%%%%%%%%%%%%%%%%%%%%%%%%%%%%%%%%%%%%%%%%%

          
 %%%%%%%%%%%%%%%%%%%%%%%%%%%%%%%%%%%%%%%%%%%%%%%%%%%%%%%%%%%%%%%%%%%%%%%%%%%%%%%
   \coordinate (currenty) at ($(box.south)-(yshift)$);

    \node[anchor=north, rounded corners=15pt, fill=titledrawcolor] (qrw) at ($(currenty) + (-34,0)$) {\includegraphics[width=8cm]{qrwebsite}};
    \node[anchor=north, rounded corners=15pt, fill=titledrawcolor] (qrgit) at ($(currenty) + (34,0)$) {\includegraphics[width=8cm]{QRvawiswow}};
    \node[anchor=south] (web) at (qrw.north) {\textbf{\color{colorone!80!black}{Website}}};
    \node[anchor=south] (git) at (qrgit.north) {\textbf{\color{colorone!80!black}{R and \LaTeX code}}};
		

		
%%%%%%%%%%%%%%%%%%%%%%%%%%%%%%%%%%%%%%%%%%%%%		
   \Endblock{($(web.east)+(2,-4)$)}{57}{\hspace{25cm}\color{colorone}{\textsc{Co-authors:}}}{ 
   Michael Morrissey, Josephine Pemberton, Tim Clutton-Brock, Marco Festa-Bianchet, Andrew McAdam, Stan Boutin, Anne Charmantier,
C\'eline Teplistky, Christophe de Franceschi, Erik Postma, Glauco Camenisch,
Marcel E. Visser, Ben Sheldon, Simon Evans, Lars Gustafsson,
Jane Reid, Matthew Wolack, Andrew Cockburn \\

{\color{colorone}{\**} Fisher's theorem relies on stringent assumptions, or alternatively on quite a specific meaning of evolution: }
  }
	
\end{tikzpicture}
  \end{document}
